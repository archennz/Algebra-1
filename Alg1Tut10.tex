\documentclass[11pt,twoside, a4paper]{report}
\usepackage{amsmath,amssymb,amsthm,array,tikz,fancyhdr,pgfplots}
\usepackage[all]{xy}
\setcounter{tocdepth}{2}
%\pagestyle{fancy}
%\fancyhf{}
%\fancyhead[LE,LO]{\leftmark}
%\fancyhead[RE,LO]{\rightmark}
%\fancyfoot[CE,CO]{\leftmark}
%\fancyfoot[LE,LO]{\thepage}
\textwidth 130mm
\textheight 220mm
\hoffset 3.5cm
\oddsidemargin -3.0cm
\evensidemargin -3.0cm
\topmargin -.5 cm
\linespread{1.2}



\theoremstyle{plain}
\newtheorem{Theorem}{Theorem}[section]
\newtheorem{Proposition}{THEOREM}[Theorem]
\newtheorem{Lemma}{Lemma}[Theorem]
\newtheorem{corollary}{Corollary}[Theorem]


\theoremstyle{definition}
\newtheorem{Definition}[Theorem]{Definition}
\newtheorem{exmp}[Theorem]{Example}
\newtheorem{obs}[Theorem]{Observation}
\newtheorem{rmrk}[Theorem]{Remark}

%QED - End of proof symbol 
\renewcommand\qedsymbol{$\blacksquare$}

%Command Shortcuts
\newcommand{\cl}{\textnormal{cl}} %Conjgacy Class.
\newcommand{\C}{\textnormal{C}}   %Centralizer.
\newcommand{\orb}{\textnormal{orb}} %Orbit of a point.
\newcommand{\stab}{\textnormal{stab}} %Stablizer of a point.
\newcommand{\Z}{\textnormal{Z}} %Center of a group.
\newcommand{\GL}{\textnormal{GL}} %General Linear Group.
\newcommand{\Ker}{\textnormal{ker}}%Kernel of a morphism.
\newcommand{\Image}{\textnormal{im}}%Image of a morphism.
\newcommand{\Coker}{\textnormal{coker}}%cokernel of a morphism.
\newcommand{\Hom}{\textnormal{Hom}} %Homomorphism group.
\newcommand{\End}{\textnormal{End}} %Homomorphisms from G to G.
\newcommand{\otr}{\otimes_{R}}%Tensor product of a ring R.
\newcommand{\otk}{\otimes_{K}}%Tensor product over a field K.
\newcommand{\otf}{\otimes_{F}}%Tensor product over a field F.
\newcommand{\ot}{\otimes}%Tensor product of elements .
\newcommand{\Spec}{\textnormal{Spec}}%Spectrum of a ring.
\newcommand{\MSpec}{\textnormal{MSpec}}%MaxSpectrum of a ring.
\newcommand{\rad}{\textnormal{rad}}%Jacobson radical.
\newcommand{\Tr}{\textnormal{Tr}}%Trace.
\newcommand{\Det}{\textnormal{Det}}%Determinant.
\newcommand{\dd}{\textnormal{d}}%Differential.
\newcommand{\kahAR}{\Omega_{A/R}^{1}}%Kaehler 1-forms.
\newcommand{\aff}{\mathbb{A}^{n}_{k}} %Affine n-space.
\newcommand{\polyn}{k[X_{1}, X_{2}, \cdots , X_{n}]} %Polynomial ring in n variables.
\newcommand{\Gal}{\textnormal{Gal}} %Galois group of E/F.
\newcommand{\surhom}{\twoheadrightarrow}%Arrow representing surjective map.
\newcommand{\Gel}{\textnormal{Gel}}%Gelfand Transformation .
\newcommand{\injmor}{\hookrightarrow}%Injective morphism arrow.
\newcommand{\surmor}{\twoheadrightarrow}%Surjective morphism arrow.
\newcommand{\algintk}{\mathcal{O}_{K}}%Algebraic integers in K.
\newcommand{\support}{\textnormal{Supp}}%Support of a section.
\newcommand{\finetk}{{\bf FEt$_{k}$}}% Category of FinteEtalekAlgebras.
\newcommand{\finetks}{{\bf FEt$_{k_{sep}}$}}% Category of FinteEtalekAlgebras.
\newcommand{\finetr}{{\bf FEt$_{R}$}}% Category of FinteEtaleRAlgebras.
\newcommand{\finets}{{\bf FEt$_{\bf Semi}$}}% Category of FinteEtaleSemiAlgebras.
\newcommand{\fingset}{{\bf FinGset }}%Category of Finite G sets.
\newcommand{\posreal}{\mathbb{R}_{+}}%Positive reals.
\newcommand{\otpr}{\otimes_{\mathbb{R}_{+}}}%Tensor over positive reals.
\newcommand{\bool}{\mathbb{B}}%Boolean semiring. 
\newcommand{\Aut}{\textnormal{Aut}}%Automorphism
\newcommand{\Inn}{\textnormal{Inn}}%InnerAutomorphism
\newcommand{\ev}{\textnormal{ev}}%evaluation map of R-algebras







\begin{document}

\begin{center}
 \noindent\makebox[\linewidth]{\rule{14cm}{1.5pt}} 
{\bf Algebra 1: Tutorial 10 }
 \noindent\makebox[\linewidth]{\rule{14cm}{1.5pt}}  
 \noindent\makebox[\linewidth]{\rule{14cm}{3pt}}
\end{center}

\noindent When you answer these questions practise your proof writing.\\
  {\bf Be a group, ring, or module.}
  
 
\begin{center}
{\bf Question 1: Presentation Matrices}
\end{center}

What is the presentation matrix of the following abelian groups:
\begin{itemize}
\item The abelian group generated by $x,y$, with the single relation $19x+13y=0$;
\item The abelian group generated by $x,y,z$, with the single relation $19x+13y=0$.
\end{itemize}


\begin{center}
{\bf Question 2: Presentation Matrices}
\end{center}

Identify the $\mathbb{Z}$-module presented by the following presentation matrices:
\[ \begin{bmatrix}
2 & 5 \\
4 & 10
\end{bmatrix};\qquad
\begin{bmatrix}
2 & -6 & 0 \\
-6 & 12 & 0
\end{bmatrix}. \]

\smallskip


\begin{center}
{\bf Question 3: Cyclic Modules}
\end{center}

A module is said to be \textit{cyclic} if it can be generated by exactly one element. Prove that a cyclic module is isomorphic to $R/(a)$ for some principal ideal $(a)$.


\begin{center}
{\bf Question 4: Torsion submodules}
\end{center}

Let $M$ be a $\mathbb{Z}$-module. Let $T$ consist of all $x\in M$ such that there exists a $r\neq0$ in $\mathbb{Z}$ with $rx=0$. Prove that $T$ is a submodule of $M$. The set $T$ is often called the \textit{torsion submodule} of $M$. What is the torsion submodule of $\mathbb{Z}^2\oplus\mathbb{Z}/8\mathbb{Z}\oplus\mathbb{Z}/2017\mathbb{Z}$?


\begin{center}
{\bf Question 5: A Non-Noetherian Ring}
\end{center}

Consider the ring $C(\mathbb{R})$ of continuous functions $f:\mathbb{R}\to\mathbb{R}$, where addition and multiplication of functions is performed pointwise, i.e. $(f+g)(x) = f(x)+g(x)$, $(fg)(x)=f(x)g(x)$. Show that it is not a Noetherian ring.


\begin{center}
{\bf Question 6: Modules over a Quotient Ring}
\end{center}

Let $R$ be a ring, $I$ be an ideal, and $M$ be a $R$-module. Under what conditions can we make $M$ into an $R/I$-module via $(r+I)x = rx$, $r\in I$, $x\in M$?

\end{document}