\documentclass[11pt,twoside, a4paper]{report}
\usepackage{amsmath,amssymb,amsthm,array,tikz,fancyhdr,pgfplots}
\usepackage[all]{xy}
\setcounter{tocdepth}{2}
%\pagestyle{fancy}
%\fancyhf{}
%\fancyhead[LE,LO]{\leftmark}
%\fancyhead[RE,LO]{\rightmark}
%\fancyfoot[CE,CO]{\leftmark}
%\fancyfoot[LE,LO]{\thepage}
\textwidth 130mm
\textheight 220mm
\hoffset 3.5cm
\oddsidemargin -3.0cm
\evensidemargin -3.0cm
\topmargin -.5cm
\linespread{1.2}



\theoremstyle{plain}
\newtheorem{Theorem}{Theorem}[section]
\newtheorem{Proposition}{THEOREM}[Theorem]
\newtheorem{Lemma}{Lemma}[Theorem]
\newtheorem{corollary}{Corollary}[Theorem]


\theoremstyle{definition}
\newtheorem{Definition}[Theorem]{Definition}
\newtheorem{exmp}[Theorem]{Example}
\newtheorem{obs}[Theorem]{Observation}
\newtheorem{rmrk}[Theorem]{Remark}

%QED - End of proof symbol 
\renewcommand\qedsymbol{$\blacksquare$}

%Command Shortcuts
\newcommand{\cl}{\textnormal{cl}} %Conjgacy Class.
\newcommand{\C}{\textnormal{C}}   %Centralizer.
\newcommand{\orb}{\textnormal{orb}} %Orbit of a point.
\newcommand{\stab}{\textnormal{stab}} %Stablizer of a point.
\newcommand{\Z}{\textnormal{Z}} %Center of a group.
\newcommand{\GL}{\textnormal{GL}} %General Linear Group.
\newcommand{\Ker}{\textnormal{ker}}%Kernel of a morphism.
\newcommand{\Image}{\textnormal{im}}%Image of a morphism.
\newcommand{\Coker}{\textnormal{coker}}%cokernel of a morphism.
\newcommand{\Hom}{\textnormal{Hom}} %Homomorphism group.
\newcommand{\End}{\textnormal{End}} %Homomorphisms from G to G.
\newcommand{\otr}{\otimes_{R}}%Tensor product of a ring R.
\newcommand{\otk}{\otimes_{K}}%Tensor product over a field K.
\newcommand{\otf}{\otimes_{F}}%Tensor product over a field F.
\newcommand{\ot}{\otimes}%Tensor product of elements .
\newcommand{\Spec}{\textnormal{Spec}}%Spectrum of a ring.
\newcommand{\MSpec}{\textnormal{MSpec}}%MaxSpectrum of a ring.
\newcommand{\rad}{\textnormal{rad}}%Jacobson radical.
\newcommand{\Tr}{\textnormal{Tr}}%Trace.
\newcommand{\Det}{\textnormal{Det}}%Determinant.
\newcommand{\dd}{\textnormal{d}}%Differential.
\newcommand{\kahAR}{\Omega_{A/R}^{1}}%Kaehler 1-forms.
\newcommand{\aff}{\mathbb{A}^{n}_{k}} %Affine n-space.
\newcommand{\polyn}{k[X_{1}, X_{2}, \cdots , X_{n}]} %Polynomial ring in n variables.
\newcommand{\Gal}{\textnormal{Gal}} %Galois group of E/F.
\newcommand{\surhom}{\twoheadrightarrow}%Arrow representing surjective map.
\newcommand{\Gel}{\textnormal{Gel}}%Gelfand Transformation .
\newcommand{\injmor}{\hookrightarrow}%Injective morphism arrow.
\newcommand{\surmor}{\twoheadrightarrow}%Surjective morphism arrow.
\newcommand{\algintk}{\mathcal{O}_{K}}%Algebraic integers in K.
\newcommand{\support}{\textnormal{Supp}}%Support of a section.
\newcommand{\finetk}{{\bf FEt$_{k}$}}% Category of FinteEtalekAlgebras.
\newcommand{\finetks}{{\bf FEt$_{k_{sep}}$}}% Category of FinteEtalekAlgebras.
\newcommand{\finetr}{{\bf FEt$_{R}$}}% Category of FinteEtaleRAlgebras.
\newcommand{\finets}{{\bf FEt$_{\bf Semi}$}}% Category of FinteEtaleSemiAlgebras.
\newcommand{\fingset}{{\bf FinGset }}%Category of Finite G sets.
\newcommand{\posreal}{\mathbb{R}_{+}}%Positive reals.
\newcommand{\otpr}{\otimes_{\mathbb{R}_{+}}}%Tensor over positive reals.
\newcommand{\bool}{\mathbb{B}}%Boolean semiring. 
\newcommand{\Aut}{\textnormal{Aut}}%Automorphism
\newcommand{\Inn}{\textnormal{Inn}}%InnerAutomorphism
\newcommand{\ev}{\textnormal{ev}}%evaluation map of R-algebras







\begin{document}

\begin{center}
 \noindent\makebox[\linewidth]{\rule{14cm}{1.5pt}} 
{\bf Algebra 1: Tutorial 2 }
 \noindent\makebox[\linewidth]{\rule{14cm}{1.5pt}}  
 \noindent\makebox[\linewidth]{\rule{14cm}{3pt}}
\end{center}

\noindent When you answer these questions practise your proof writing.\\
  {\bf Be clear, concise, and complete.}
  
  


  
\begin{center}
{\bf Question 1: Automorphism Group}
\end{center}

Let $\Aut(G):=\{\textnormal{isomorphisms from G to G}\}$. Prove $\Aut(G)$ is a group under homomorphism composition. Isomorphisms from a group to itself are called automorphisms.


  
\begin{center}
{\bf Question 2: Isomorphisms as ``equality" }
\end{center}

Isomorphisms of groups are meant to be the notion of ``equality'' on the class of all groups. Check that the notion of isomorphism is indeed an equivalence relation on the class of all groups.

Let $G$ be a group. Prove that the relation $a \sim b$ if $b=gag^{-1}$ for some $g \in G$ is an equivalence relation. That is to say, conjugation is an equivalence relation.

  
\begin{center}
{\bf Question 3: Partial Converse of Lagrange's Theorem}
\end{center}

Find all subgroups of $\mathbb{Z}/2\mathbb{Z}$, $\mathbb{Z}/3\mathbb{Z}, \dots , \mathbb{Z}/p\mathbb{Z}$, for $p$ prime. What can you say about the converse of Lagrange's theorem if the order of $G$ is prime?

  
\begin{center}
{\bf Question 4: Further Partial Converse of Lagrange's Theorem}
\end{center}

Let $G$ be finite cyclic of order $n$. If $k|n$, then show $G$ has a subgroup of order $k$. What can you say about the converse of Lagrange's theorem for finite cyclic groups?

  
\begin{center}
{\bf Question 5: Converse of Lagrange's Theorem is False in General}
\end{center}
Let $G:=A_{4}$. What is $|A_{4}|$? Does $A_{4}$ have a subgroup of order 6? [Hint: Suppose it does, denote it $H$. What is the index $[A_{4}:H]$? How many left cosets of $H$ are there? How many elements of order 3 are in $H$?] In general, then, is the converse of Lagrange's theorem true?

  
\begin{center}
{\bf Question 6: Decomposing Groups as Products}
\end{center}

Consider the subgroups $H:=\{0,4,8\}$ and $K:=\{0,3,6,9\}$ of $\mathbb{Z}/12\mathbb{Z}$ --- why are they subgroups? Prove $HK=\mathbb{Z}/12\mathbb{Z}$, and $H\cap K = \{0\}$. What can you conclude about the structure of $\mathbb{Z}/12\mathbb{Z}$?

  
\begin{center}
{\bf Question 7: Groups in Number Theory}
\end{center}

Find all squares mod $4$. Conclude that $x^{2} - 4y^{2} = 2$ has no solutions in the integers. Group theory can help us understand number theory: it is not all about symmetry and permutations. 

  
\begin{center}
{\bf Question 8: Automorphism Example}
\end{center}

If $G$ is a finite group, under what circumstances is $\varphi: G \rightarrow G$ where $x\mapsto x^{2}$ an automorphism of $G$?

\end{document}