\documentclass[11pt,twoside, a4paper]{report}
\usepackage{amsmath,amssymb,amsthm,array,tikz,fancyhdr,pgfplots}
\usepackage[all]{xy}
\setcounter{tocdepth}{2}
%\pagestyle{fancy}
%\fancyhf{}
%\fancyhead[LE,LO]{\leftmark}
%\fancyhead[RE,LO]{\rightmark}
%\fancyfoot[CE,CO]{\leftmark}
%\fancyfoot[LE,LO]{\thepage}
\textwidth 130mm
\textheight 220mm
\hoffset 3.5cm
\oddsidemargin -3.0cm
\evensidemargin -3.0cm
\topmargin -.5cm
\linespread{1.2}



\theoremstyle{plain}
\newtheorem{Theorem}{Theorem}[section]
\newtheorem{Proposition}{THEOREM}[Theorem]
\newtheorem{Lemma}{Lemma}[Theorem]
\newtheorem{corollary}{Corollary}[Theorem]


\theoremstyle{definition}
\newtheorem{Definition}[Theorem]{Definition}
\newtheorem{exmp}[Theorem]{Example}
\newtheorem{obs}[Theorem]{Observation}
\newtheorem{rmrk}[Theorem]{Remark}

%QED - End of proof symbol 
\renewcommand\qedsymbol{$\blacksquare$}

%Command Shortcuts
\newcommand{\cl}{\textnormal{cl}} %Conjgacy Class.
\newcommand{\C}{\textnormal{C}}   %Centralizer.
\newcommand{\orb}{\textnormal{orb}} %Orbit of a point.
\newcommand{\stab}{\textnormal{stab}} %Stablizer of a point.
\newcommand{\Z}{\textnormal{Z}} %Center of a group.
\newcommand{\GL}{\textnormal{GL}} %General Linear Group.
\newcommand{\Ker}{\textnormal{ker}}%Kernel of a morphism.
\newcommand{\Image}{\textnormal{im}}%Image of a morphism.
\newcommand{\Coker}{\textnormal{coker}}%cokernel of a morphism.
\newcommand{\Hom}{\textnormal{Hom}} %Homomorphism group.
\newcommand{\End}{\textnormal{End}} %Homomorphisms from G to G.
\newcommand{\otr}{\otimes_{R}}%Tensor product of a ring R.
\newcommand{\otk}{\otimes_{K}}%Tensor product over a field K.
\newcommand{\otf}{\otimes_{F}}%Tensor product over a field F.
\newcommand{\ot}{\otimes}%Tensor product of elements .
\newcommand{\Spec}{\textnormal{Spec}}%Spectrum of a ring.
\newcommand{\MSpec}{\textnormal{MSpec}}%MaxSpectrum of a ring.
\newcommand{\rad}{\textnormal{rad}}%Jacobson radical.
\newcommand{\Tr}{\textnormal{Tr}}%Trace.
\newcommand{\Det}{\textnormal{Det}}%Determinant.
\newcommand{\dd}{\textnormal{d}}%Differential.
\newcommand{\kahAR}{\Omega_{A/R}^{1}}%Kaehler 1-forms.
\newcommand{\aff}{\mathbb{A}^{n}_{k}} %Affine n-space.
\newcommand{\polyn}{k[X_{1}, X_{2}, \cdots , X_{n}]} %Polynomial ring in n variables.
\newcommand{\Gal}{\textnormal{Gal}} %Galois group of E/F.
\newcommand{\surhom}{\twoheadrightarrow}%Arrow representing surjective map.
\newcommand{\Gel}{\textnormal{Gel}}%Gelfand Transformation .
\newcommand{\injmor}{\hookrightarrow}%Injective morphism arrow.
\newcommand{\surmor}{\twoheadrightarrow}%Surjective morphism arrow.
\newcommand{\algintk}{\mathcal{O}_{K}}%Algebraic integers in K.
\newcommand{\support}{\textnormal{Supp}}%Support of a section.
\newcommand{\finetk}{{\bf FEt$_{k}$}}% Category of FinteEtalekAlgebras.
\newcommand{\finetks}{{\bf FEt$_{k_{sep}}$}}% Category of FinteEtalekAlgebras.
\newcommand{\finetr}{{\bf FEt$_{R}$}}% Category of FinteEtaleRAlgebras.
\newcommand{\finets}{{\bf FEt$_{\bf Semi}$}}% Category of FinteEtaleSemiAlgebras.
\newcommand{\fingset}{{\bf FinGset }}%Category of Finite G sets.
\newcommand{\posreal}{\mathbb{R}_{+}}%Positive reals.
\newcommand{\otpr}{\otimes_{\mathbb{R}_{+}}}%Tensor over positive reals.
\newcommand{\bool}{\mathbb{B}}%Boolean semiring. 
\newcommand{\Aut}{\textnormal{Aut}}%Automorphism
\newcommand{\Inn}{\textnormal{Inn}}%InnerAutomorphism
\newcommand{\ev}{\textnormal{ev}}%evaluation map of R-algebras







\begin{document}

\begin{center}
 \noindent\makebox[\linewidth]{\rule{14cm}{1.5pt}} 
{\bf Algebra 1: Tutorial 1 }
 \noindent\makebox[\linewidth]{\rule{14cm}{1.5pt}}  
 \noindent\makebox[\linewidth]{\rule{14cm}{3pt}}
\end{center}

\noindent When you answer these questions practise your proof writing.\\
  {\bf Be clear, concise, and complete.}
  
  
\begin{center}
{\bf Question 1: Examples of Subgroups}
\end{center}


Which of the following $H$ are subgroups of the given group $G$: 

\begin{itemize}
\item[(a)] $G:= \mathbb{C}^{*}$ and $H:=\{ \pm 1, \pm i\}$ 

\item[(b)] $G:= \mathbb{Z}$ and $H:=\mathbb{N}$

\item[(c)] $G:=\textnormal{GL}_{2}(\mathbb{C})$ and $H:=\textnormal{SL}_{2}(\mathbb{C}):=\{A \in \textnormal{GL}_{2}(\mathbb{C}) \ | \ \det(A)=1 \}$

\item[(d)] $G:= \mathbb{Z}$ and $H:=\{0\}$
\end{itemize}

\begin{center}
{\bf Question 2: New Groups From Old}
\end{center}

%NOTE: Joan does not want topic from the future of the course in the tutorial; just let them think about the topics they're already introduced to. 

Let $G_{1}$ and $G_{2}$ be groups. Show that there is a ``natural" group structure on the set theoretic product $G_{1}\times G_{2}$. If $|G_{1}|=n$ and $|G_{2}|=m$, what is $|G_{1}\times G_{2}|$?

\begin{center}
{\bf Question 3: Permutations}
\end{center}


Let $ \sigma = \bigl(\begin{smallmatrix}
    1 & 2 & 3 & 4 & 5 & 6 & 7 \\
    2 & 4 & 7 & 3 & 1 & 6 & 5
  \end{smallmatrix}\bigr)$ be a permutation on the set $\{1, \dots , 7\}$ 
  
  \begin{itemize}
  \item[(a)] write $\sigma$ in cycle notation
  \item[(b)] write $\sigma$ as composition of transpositions. What is sgn$(\sigma)$?
  \end{itemize}


\begin{center}
{\bf Question 4: Classification of Subgroups of $\mathbb{Z}$}
\end{center}

Give an example of a subgroup of $\mathbb{Z}$. Is it cyclic? Can you list \emph{all} subgroups of $\mathbb{Z}$? Are they all cyclic?

\begin{center}
{\bf Question 5: Group Homomorphism Example}
\end{center}

What is a homomorphism of groups? What is an isomorphism? Find a homomorphism 
$$\varphi: \mathbb{C}^{*} \rightarrow \textnormal{GL}_{2}(\mathbb{R})$$
Is the homomorphism (a) injective? (b) surjective? (c) an isomorphism? [Hint: In order to construct a homomorphism, think about where $1$ and $i$ must be mapped to]

\begin{center}
{\bf Question 6: Cosets}
\end{center}

Let $H:=4\mathbb{Z} \subseteq \mathbb{Z}$ be a subgroup of integers. 

\begin{itemize}
\item[(a)] are 3 and 7 in the same coset? 
\item[(b)] are 3 and 6 in the same coset?
\item[(c)] are 0, 4 and 24 in the same coset?
\end{itemize}

%NOTE: Joan does not want topic from the future of the course in the tutorial; just let them think about the topics they're already introduced to. 


%How might you construct an operation on the set of all cosets of a subgroup $H$ of $G$? That is to say, given 2 cosets, say, $3+4\mathbb{Z}$ and $0+4\mathbb{Z}$, how can you get another one? This will be covered more precisely in the lectures, but it is worth thinking about as it is a very important construction. 


\end{document}