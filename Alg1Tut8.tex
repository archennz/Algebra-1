\documentclass[11pt,twoside, a4paper]{report}
\usepackage{amsmath,amssymb,amsthm,array,tikz,fancyhdr,pgfplots}
\usepackage[all]{xy}
\setcounter{tocdepth}{2}
%\pagestyle{fancy}
%\fancyhf{}
%\fancyhead[LE,LO]{\leftmark}
%\fancyhead[RE,LO]{\rightmark}
%\fancyfoot[CE,CO]{\leftmark}
%\fancyfoot[LE,LO]{\thepage}
\textwidth 130mm
\textheight 220mm
\hoffset 3.5cm
\oddsidemargin -3.0cm
\evensidemargin -3.0cm
\topmargin -.5 cm
\linespread{1.2}



\theoremstyle{plain}
\newtheorem{Theorem}{Theorem}[section]
\newtheorem{Proposition}{THEOREM}[Theorem]
\newtheorem{Lemma}{Lemma}[Theorem]
\newtheorem{corollary}{Corollary}[Theorem]


\theoremstyle{definition}
\newtheorem{Definition}[Theorem]{Definition}
\newtheorem{exmp}[Theorem]{Example}
\newtheorem{obs}[Theorem]{Observation}
\newtheorem{rmrk}[Theorem]{Remark}

%QED - End of proof symbol 
\renewcommand\qedsymbol{$\blacksquare$}

%Command Shortcuts
\newcommand{\cl}{\textnormal{cl}} %Conjgacy Class.
\newcommand{\C}{\textnormal{C}}   %Centralizer.
\newcommand{\orb}{\textnormal{orb}} %Orbit of a point.
\newcommand{\stab}{\textnormal{stab}} %Stablizer of a point.
\newcommand{\Z}{\textnormal{Z}} %Center of a group.
\newcommand{\GL}{\textnormal{GL}} %General Linear Group.
\newcommand{\Ker}{\textnormal{ker}}%Kernel of a morphism.
\newcommand{\Image}{\textnormal{im}}%Image of a morphism.
\newcommand{\Coker}{\textnormal{coker}}%cokernel of a morphism.
\newcommand{\Hom}{\textnormal{Hom}} %Homomorphism group.
\newcommand{\End}{\textnormal{End}} %Homomorphisms from G to G.
\newcommand{\otr}{\otimes_{R}}%Tensor product of a ring R.
\newcommand{\otk}{\otimes_{K}}%Tensor product over a field K.
\newcommand{\otf}{\otimes_{F}}%Tensor product over a field F.
\newcommand{\ot}{\otimes}%Tensor product of elements .
\newcommand{\Spec}{\textnormal{Spec}}%Spectrum of a ring.
\newcommand{\MSpec}{\textnormal{MSpec}}%MaxSpectrum of a ring.
\newcommand{\rad}{\textnormal{rad}}%Jacobson radical.
\newcommand{\Tr}{\textnormal{Tr}}%Trace.
\newcommand{\Det}{\textnormal{Det}}%Determinant.
\newcommand{\dd}{\textnormal{d}}%Differential.
\newcommand{\kahAR}{\Omega_{A/R}^{1}}%Kaehler 1-forms.
\newcommand{\aff}{\mathbb{A}^{n}_{k}} %Affine n-space.
\newcommand{\polyn}{k[X_{1}, X_{2}, \cdots , X_{n}]} %Polynomial ring in n variables.
\newcommand{\Gal}{\textnormal{Gal}} %Galois group of E/F.
\newcommand{\surhom}{\twoheadrightarrow}%Arrow representing surjective map.
\newcommand{\Gel}{\textnormal{Gel}}%Gelfand Transformation .
\newcommand{\injmor}{\hookrightarrow}%Injective morphism arrow.
\newcommand{\surmor}{\twoheadrightarrow}%Surjective morphism arrow.
\newcommand{\algintk}{\mathcal{O}_{K}}%Algebraic integers in K.
\newcommand{\support}{\textnormal{Supp}}%Support of a section.
\newcommand{\finetk}{{\bf FEt$_{k}$}}% Category of FinteEtalekAlgebras.
\newcommand{\finetks}{{\bf FEt$_{k_{sep}}$}}% Category of FinteEtalekAlgebras.
\newcommand{\finetr}{{\bf FEt$_{R}$}}% Category of FinteEtaleRAlgebras.
\newcommand{\finets}{{\bf FEt$_{\bf Semi}$}}% Category of FinteEtaleSemiAlgebras.
\newcommand{\fingset}{{\bf FinGset }}%Category of Finite G sets.
\newcommand{\posreal}{\mathbb{R}_{+}}%Positive reals.
\newcommand{\otpr}{\otimes_{\mathbb{R}_{+}}}%Tensor over positive reals.
\newcommand{\bool}{\mathbb{B}}%Boolean semiring. 
\newcommand{\Aut}{\textnormal{Aut}}%Automorphism
\newcommand{\Inn}{\textnormal{Inn}}%InnerAutomorphism
\newcommand{\ev}{\textnormal{ev}}%evaluation map of R-algebras







\begin{document}

\begin{center}
 \noindent\makebox[\linewidth]{\rule{14cm}{1.5pt}} 
{\bf Algebra 1: Tutorial 8 }
 \noindent\makebox[\linewidth]{\rule{14cm}{1.5pt}}  
 \noindent\makebox[\linewidth]{\rule{14cm}{3pt}}
\end{center}

\noindent When you answer these questions practise your proof writing.\\
  {\bf Be clear, concise, and complete.}
  
  
\begin{center}
{\bf Question 1: Idempotents are Zero Divisors}
\end{center}

Recall that an idempotent of a ring $R$ is an element $e \in R$ such that $e^{2}=e$. The identities $0$ and $1$ of a ring are always idempotents, we call these \emph{trivial idempotents}. Prove all non-trivial idempotents in a ring $R$ are zero divisors. Prove that an integral domain can't be decomposed as a product of smaller rings. 



\begin{center}
{\bf Question 2: Idempotents Break Rings Up}
\end{center} 

If $A$ and $B$ are rings, then show $(a,b) \in A\times B$ is an idempotent if and only if $a$ and $b$ are idempotents in $A$ and $B$ respectively. Find all idempotents in (i) $\mathbb{R}^{2}$ (ii) $\mathbb{Z}/3\mathbb{Z} \times \mathbb{Z}/6\mathbb{Z}$ (iii) $\mathbb{Z}\times\mathbb{Q}$

\begin{center}
{\bf Question 3: Product Rings Example I}
\end{center}

Decompose $\mathbb{Z}/6\mathbb{Z}$ as a product of rings. Decompose $\mathbb{Z}/12\mathbb{Z}$ as a product of rings. 


\begin{center}
{\bf Question 4: Ideal of Vanishing are Maximal}
\end{center}

Recall the ideal $I_{a}:=\{ f \in \mathbb{C}[x] \ | \ f(a) = 0 \} = \langle x-a \rangle \subseteq \mathbb{C}[x]$ from the last few tutorials. Prove this ideal is a maximal ideal. 

This proves there is a bijection between points of $\mathbb{C}$ and maximal ideals of $\mathbb{C}[x]$: this is the beginning point of algebraic geometry. 



\begin{center}
{\bf Question 5: Quotient Ring Examples I}
\end{center} 

Prove $f(x)=x^{2}-2$ is irreducible over $\mathbb{Q}$. As a consequence prove $\mathbb{Q}[x]/\langle x^{2}-2 \rangle$ is a field. Identify this field with a subfield of $\mathbb{R}$. Do the same with (i) $g(x)=x^{2} -3$ (ii) $h(x) = x^{3} - 2$. One can't always realise such quotient fields as subfields of $\mathbb{R}$: for example the quotient $\mathbb{Q}[x]/\langle  x^{2} + 1\rangle$ can't be imbedded into $\mathbb{R}$ --- why? --- instead, realise this as a subfield of $\mathbb{C}$. 

Prove $\mathbb{Z}/3\mathbb{Z}[x]/\langle x^{2} + 1 \rangle$ is a field. How many elements does it have?




\end{document}