\documentclass[11pt,twoside, a4paper]{report}
\usepackage{amsmath,amssymb,amsthm,array,tikz,fancyhdr,pgfplots}
\usepackage[all]{xy}
\setcounter{tocdepth}{2}
%\pagestyle{fancy}
%\fancyhf{}
%\fancyhead[LE,LO]{\leftmark}
%\fancyhead[RE,LO]{\rightmark}
%\fancyfoot[CE,CO]{\leftmark}
%\fancyfoot[LE,LO]{\thepage}
\textwidth 130mm
\textheight 220mm
\hoffset 3.5cm
\oddsidemargin -3.0cm
\evensidemargin -3.0cm
\topmargin -.5cm
\linespread{1.2}



\theoremstyle{plain}
\newtheorem{Theorem}{Theorem}[section]
\newtheorem{Proposition}{THEOREM}[Theorem]
\newtheorem{Lemma}{Lemma}[Theorem]
\newtheorem{corollary}{Corollary}[Theorem]


\theoremstyle{definition}
\newtheorem{Definition}[Theorem]{Definition}
\newtheorem{exmp}[Theorem]{Example}
\newtheorem{obs}[Theorem]{Observation}
\newtheorem{rmrk}[Theorem]{Remark}

%QED - End of proof symbol 
\renewcommand\qedsymbol{$\blacksquare$}

%Command Shortcuts
\newcommand{\cl}{\textnormal{cl}} %Conjgacy Class.
\newcommand{\C}{\textnormal{C}}   %Centralizer.
\newcommand{\orb}{\textnormal{orb}} %Orbit of a point.
\newcommand{\stab}{\textnormal{stab}} %Stablizer of a point.
\newcommand{\Z}{\textnormal{Z}} %Center of a group.
\newcommand{\GL}{\textnormal{GL}} %General Linear Group.
\newcommand{\Ker}{\textnormal{ker}}%Kernel of a morphism.
\newcommand{\Image}{\textnormal{im}}%Image of a morphism.
\newcommand{\Coker}{\textnormal{coker}}%cokernel of a morphism.
\newcommand{\Hom}{\textnormal{Hom}} %Homomorphism group.
\newcommand{\End}{\textnormal{End}} %Homomorphisms from G to G.
\newcommand{\otr}{\otimes_{R}}%Tensor product of a ring R.
\newcommand{\otk}{\otimes_{K}}%Tensor product over a field K.
\newcommand{\otf}{\otimes_{F}}%Tensor product over a field F.
\newcommand{\ot}{\otimes}%Tensor product of elements .
\newcommand{\Spec}{\textnormal{Spec}}%Spectrum of a ring.
\newcommand{\MSpec}{\textnormal{MSpec}}%MaxSpectrum of a ring.
\newcommand{\rad}{\textnormal{rad}}%Jacobson radical.
\newcommand{\Tr}{\textnormal{Tr}}%Trace.
\newcommand{\Det}{\textnormal{Det}}%Determinant.
\newcommand{\dd}{\textnormal{d}}%Differential.
\newcommand{\kahAR}{\Omega_{A/R}^{1}}%Kaehler 1-forms.
\newcommand{\aff}{\mathbb{A}^{n}_{k}} %Affine n-space.
\newcommand{\polyn}{k[X_{1}, X_{2}, \cdots , X_{n}]} %Polynomial ring in n variables.
\newcommand{\Gal}{\textnormal{Gal}} %Galois group of E/F.
\newcommand{\surhom}{\twoheadrightarrow}%Arrow representing surjective map.
\newcommand{\Gel}{\textnormal{Gel}}%Gelfand Transformation .
\newcommand{\injmor}{\hookrightarrow}%Injective morphism arrow.
\newcommand{\surmor}{\twoheadrightarrow}%Surjective morphism arrow.
\newcommand{\algintk}{\mathcal{O}_{K}}%Algebraic integers in K.
\newcommand{\support}{\textnormal{Supp}}%Support of a section.
\newcommand{\finetk}{{\bf FEt$_{k}$}}% Category of FinteEtalekAlgebras.
\newcommand{\finetks}{{\bf FEt$_{k_{sep}}$}}% Category of FinteEtalekAlgebras.
\newcommand{\finetr}{{\bf FEt$_{R}$}}% Category of FinteEtaleRAlgebras.
\newcommand{\finets}{{\bf FEt$_{\bf Semi}$}}% Category of FinteEtaleSemiAlgebras.
\newcommand{\fingset}{{\bf FinGset }}%Category of Finite G sets.
\newcommand{\posreal}{\mathbb{R}_{+}}%Positive reals.
\newcommand{\otpr}{\otimes_{\mathbb{R}_{+}}}%Tensor over positive reals.
\newcommand{\bool}{\mathbb{B}}%Boolean semiring. 
\newcommand{\Aut}{\textnormal{Aut}}%Automorphism
\newcommand{\Inn}{\textnormal{Inn}}%InnerAutomorphism
\newcommand{\ev}{\textnormal{ev}}%evaluation map of R-algebras







\begin{document}

\begin{center}
 \noindent\makebox[\linewidth]{\rule{14cm}{1.5pt}} 
{\bf Algebra 1: Tutorial 6 }
 \noindent\makebox[\linewidth]{\rule{14cm}{1.5pt}}  
 \noindent\makebox[\linewidth]{\rule{14cm}{3pt}}
\end{center}

\noindent When you answer these questions practise your proof writing.\\
  {\bf Be clear, concise, and complete.}
  
  
\begin{center}
{\bf Question 1: Underlying Group Structures in Rings}
\end{center}

Pick your favourite group. Can you find a ``natural" ring structure on it? [Recall that a ring is a commutative group under addition with another operation, multiplication. So can you see a multiplication on your favourite group which induces a ring structure? Perhaps the group already has multiplication, in which case you need the additive structure.]


  
\begin{center}
{\bf Question 2: Examples}
\end{center}

Give an example of 
\begin{itemize} 

\item a ring which is not a field, but has a non-trivial (not $\pm 1$) unit.
\item a ring in which all non-zero elements are units. 
\item an ideal of $\mathbb{Z}$. Is it principal?
\item a ring with no zero divisors and only the trivial (not $\pm 1$) units.
\end{itemize}
  
\begin{center}
{\bf Question 3: Non-Commutative Example}
\end{center}

Some mathematicians prefer to relax the definition of a ring a little bit. Many do not require the \emph{multiplication} to be commutative. Can you give an example of such a \emph{non-commutative ring}?
  
\begin{center}
{\bf Question 4: Kernels are Ideals}
\end{center} 

If $\phi: R \rightarrow R'$ is a ring homomorphism, prove $\ker(\phi)$ is an ideal of $R$.

  
\begin{center}
{\bf Question 5: Homomorphisms Determined by Generators}
\end{center}

Give a homomorphism $\phi: \mathbb{R}[x] \rightarrow \mathbb{R}$ which fixes elements of $\mathbb{R}$. [This can be done by applying the \emph{substitution principle} with the identity map $i:\mathbb{R}\rightarrow \mathbb{R}$]

  
\begin{center}
{\bf Question 6: Units and Trivial Ideals}
\end{center}

If $u\in I$ is a unit in an ideal $I\subseteq R$, what can you say about $I$?

  
\begin{center}
{\bf Question 7: Primes Factor in Extensions}
\end{center} 

If $i = \sqrt{-1}$, show $\mathbb{Z}[i]:=\{ a + bi \ | \ a, b \in \mathbb{Z}\}$ is a ring with the standard multiplication of complex numbers. Can you factor 2 in $\mathbb{Z}[i]$?

  
\begin{center}
{\bf Question 8: Zero Divisors }
\end{center}

For which $n$ does $\mathbb{Z}/n\mathbb{Z}$ contain zero-divisors? When is $\mathbb{Z}/n\mathbb{Z}$ a field?

  
\begin{center}
{\bf Question 9: Ideal of Vanishing}
\end{center}

Let $a \in \mathbb{C}$ and define the set $I_{a}:= \{ f \in \mathbb{C}[x] \ | \ f(a) = 0 \} \subseteq \mathbb{C}[x]$. Prove $I_{a}$ is an ideal in $\mathbb{C}[x]$. Moreover, prove it is principal. 


\end{document}