\documentclass[11pt,twoside, a4paper]{report}
\usepackage{amsmath,amssymb,amsthm,array,tikz,fancyhdr,pgfplots}
\usepackage[all]{xy}
\setcounter{tocdepth}{2}
%\pagestyle{fancy}
%\fancyhf{}
%\fancyhead[LE,LO]{\leftmark}
%\fancyhead[RE,LO]{\rightmark}
%\fancyfoot[CE,CO]{\leftmark}
%\fancyfoot[LE,LO]{\thepage}
\textwidth 130mm
\textheight 220mm
\hoffset 3.5cm
\oddsidemargin -3.0cm
\evensidemargin -3.0cm
\topmargin -.5 cm
\linespread{1.2}



\theoremstyle{plain}
\newtheorem{Theorem}{Theorem}[section]
\newtheorem{Proposition}{THEOREM}[Theorem]
\newtheorem{Lemma}{Lemma}[Theorem]
\newtheorem{corollary}{Corollary}[Theorem]


\theoremstyle{definition}
\newtheorem{Definition}[Theorem]{Definition}
\newtheorem{exmp}[Theorem]{Example}
\newtheorem{obs}[Theorem]{Observation}
\newtheorem{rmrk}[Theorem]{Remark}

%QED - End of proof symbol 
\renewcommand\qedsymbol{$\blacksquare$}

%Command Shortcuts
\newcommand{\cl}{\textnormal{cl}} %Conjgacy Class.
\newcommand{\C}{\textnormal{C}}   %Centralizer.
\newcommand{\orb}{\textnormal{orb}} %Orbit of a point.
\newcommand{\stab}{\textnormal{stab}} %Stablizer of a point.
\newcommand{\Z}{\textnormal{Z}} %Center of a group.
\newcommand{\GL}{\textnormal{GL}} %General Linear Group.
\newcommand{\Ker}{\textnormal{ker}}%Kernel of a morphism.
\newcommand{\Image}{\textnormal{im}}%Image of a morphism.
\newcommand{\Coker}{\textnormal{coker}}%cokernel of a morphism.
\newcommand{\Hom}{\textnormal{Hom}} %Homomorphism group.
\newcommand{\End}{\textnormal{End}} %Homomorphisms from G to G.
\newcommand{\otr}{\otimes_{R}}%Tensor product of a ring R.
\newcommand{\otk}{\otimes_{K}}%Tensor product over a field K.
\newcommand{\otf}{\otimes_{F}}%Tensor product over a field F.
\newcommand{\ot}{\otimes}%Tensor product of elements .
\newcommand{\Spec}{\textnormal{Spec}}%Spectrum of a ring.
\newcommand{\MSpec}{\textnormal{MSpec}}%MaxSpectrum of a ring.
\newcommand{\rad}{\textnormal{rad}}%Jacobson radical.
\newcommand{\Tr}{\textnormal{Tr}}%Trace.
\newcommand{\Det}{\textnormal{Det}}%Determinant.
\newcommand{\dd}{\textnormal{d}}%Differential.
\newcommand{\kahAR}{\Omega_{A/R}^{1}}%Kaehler 1-forms.
\newcommand{\aff}{\mathbb{A}^{n}_{k}} %Affine n-space.
\newcommand{\polyn}{k[X_{1}, X_{2}, \cdots , X_{n}]} %Polynomial ring in n variables.
\newcommand{\Gal}{\textnormal{Gal}} %Galois group of E/F.
\newcommand{\surhom}{\twoheadrightarrow}%Arrow representing surjective map.
\newcommand{\Gel}{\textnormal{Gel}}%Gelfand Transformation .
\newcommand{\injmor}{\hookrightarrow}%Injective morphism arrow.
\newcommand{\surmor}{\twoheadrightarrow}%Surjective morphism arrow.
\newcommand{\algintk}{\mathcal{O}_{K}}%Algebraic integers in K.
\newcommand{\support}{\textnormal{Supp}}%Support of a section.
\newcommand{\finetk}{{\bf FEt$_{k}$}}% Category of FinteEtalekAlgebras.
\newcommand{\finetks}{{\bf FEt$_{k_{sep}}$}}% Category of FinteEtalekAlgebras.
\newcommand{\finetr}{{\bf FEt$_{R}$}}% Category of FinteEtaleRAlgebras.
\newcommand{\finets}{{\bf FEt$_{\bf Semi}$}}% Category of FinteEtaleSemiAlgebras.
\newcommand{\fingset}{{\bf FinGset }}%Category of Finite G sets.
\newcommand{\posreal}{\mathbb{R}_{+}}%Positive reals.
\newcommand{\otpr}{\otimes_{\mathbb{R}_{+}}}%Tensor over positive reals.
\newcommand{\bool}{\mathbb{B}}%Boolean semiring. 
\newcommand{\Aut}{\textnormal{Aut}}%Automorphism
\newcommand{\Inn}{\textnormal{Inn}}%InnerAutomorphism
\newcommand{\ev}{\textnormal{ev}}%evaluation map of R-algebras







\begin{document}

\begin{center}
 \noindent\makebox[\linewidth]{\rule{14cm}{1.5pt}} 
{\bf Algebra 1: Tutorial 9 }
 \noindent\makebox[\linewidth]{\rule{14cm}{1.5pt}}  
 \noindent\makebox[\linewidth]{\rule{14cm}{3pt}}
\end{center}

\noindent When you answer these questions practise your proof writing.\\
  {\bf Be clear, concise, and complete.}
  
  
\begin{center}
{\bf Question 1: Examples of Modules}
\end{center}

Give an example of a: 

\begin{itemize}
\item[$\cdot$] $\mathbb{Z}$-module
\item[$\cdot$] $\mathbb{Z}/4\mathbb{Z}$-module
\item[$\cdot$] sub-$\mathbb{Z}$-module of the $\mathbb{Z}$-module $\mathbb{Z}$
\item[$\cdot$] $\mathbb{C}$-module
\item[$\cdot$]  sub-$\mathbb{R}$-module of the $\mathbb{R}$-module $\mathbb{C}$.
\item[$\cdot$] non-vector space free module. 

\end{itemize}



\begin{center}
{\bf Question 2: Not All Modules are Free}
\end{center} 

While modules are the natural generalisation of vector spaces over fields to ``vector spaces" over general rings, some times our intuitions do \emph{not} carry over: 
\begin{itemize} 
\item[$\cdot$] Give a module without a basis i.e. a non-free module.
\item[$\cdot$] Give a finite (as a set) module. 
\end{itemize}


\begin{center}
{\bf Question 3: Torsion}
\end{center}

One of the interesting properties that general modules have which obstructs our intuitions from vector spaces (modules over fields) is \emph{torsion}: a non-trivial element $x \in M$ of an $R$-module $M$ is said to be a \emph{torsion element} if there exists a non-trivial element $r \in R$ such that $rm=0 \in M$. Give a module with non-trivial torsion.


\begin{center}
{\bf Question 4: Modules as Endomorphisms of Groups}
\end{center}

[Warning: non-commutative rings ahead] Prove that giving an $R$-module structure on $M$ is equivalent to defining a ring homomorphism $\varphi: R \rightarrow \textnormal{End}(M)$. The ring on the right is the \emph{endomorphism} ring of $M$, which is the (in general, non-commutative) ring of group homomorphisms from $M$ to itself. 

\begin{center}
{\bf Question 5: Modules in Number Theory}
\end{center}

The Abelian group $\mathbb{Z}[\sqrt{2}]:=\{ a + b \sqrt{2} \ | \ a,b \in \mathbb{Z}\}$ is a $\mathbb{Z}$ module. Prove it is free. What does it look like as a sub-set of the plane? These $\mathbb{Z}$-modules play an important part in algebraic number theory, and solving Diophantine equations. In particular their \emph{geometric} properties tell us a lot! 


\begin{center}
{\bf Question 6: Linear Algebra of Commutative Rings}
\end{center}

Diagonalise, into a matrix with integer entries, the following matrices, and determine the matrices which diagonalise them.
$$A:= \left( \begin{array}{cc}
3 & 1  \\
-1 & 2  \\
 \end{array} \right) $$
 
$$B:= \left( \begin{array}{ccc}
3 & 1 & -4 \\
2 & -3 & 1 \\
-4 & 6 & 2 \end{array} \right)$$


\end{document}